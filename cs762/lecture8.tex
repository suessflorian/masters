\documentclass{article}
\usepackage{hyperref}
\begin{document}
(15th March, Week 3)
\section{Comparing Models - (Which one is better?)}
We can compare models with different blunt metrics like accuracy etc... we would like to rule out the possibility of tainted results by {\em chance}. We do this concretely by incorporating t-tests to provide a statistical measure against a null hypothesis.

So considering models $M_1$ and $M_2$. That is two models. We perform a 10-fold cross validation, where respectively each iteration result populates accuracy samples $A_{M_1}$ and $A_{M_2}$. Hence $|A_{M_1}| = |A_{M_2}| = 10$. We then perform an independant sample t-test (null hypothesis is that $A_{M_1}$ differs not to $A_{M_2}$).

{\em NOTE: we need to go back to the notes here and copy down all the manual formulas for calculating those t-test values etc, didn\'t bother since I would just use sci py package and just perform a significance test that way...}

\subsection{Alterantive measures}
And keep in mind that {\bf tree models are not always the best}. How does the model perform when certain features are not present? Does the model still predict well? Which features are critical etc... The \href{https://en.wikipedia.org/wiki/No_free_lunch_theorem}{no free lunch theorem}.

\end{document}
