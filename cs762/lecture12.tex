\documentclass{article}
\usepackage{hyperref}
\begin{document}
(24th March 2023)
\section*{Ensemble Methods Continued}
Last day of this legendary lecturer Jörg Simon Wicker.

\subsection*{XGBoost}
Recently (~ 6yrs) popularised. Uses regularised regression trees. Need absolutely consult the set of four videos covered by \href{https://www.youtube.com/watch?v=OtD8wVaFm6E}{statquest}.

\subsubsection*{Pre-requisite; Gradient Boost Regression}
Will use a \href{https://www.youtube.com/watch?v=3CC4N4z3GJc}{four part video series from statquest} to digest what is known as a pre-requisite. Gradient boost can be used for {\bf regression} and classification. Similar to last lectures topic of {\bf Adaboost}.

Compared a lot to Adaboost, we will see a generation of trees of fixed size (so larger than stump that is the case for Adaboost), each based/influence on the previous tree\'s performance. So rate of error.

Gradient boost is a configurable method...

\begin{itemize}
	\item Operates on Decision Trees
	\item Assumes familiarity with the bias-variance tradeoff
\end{itemize}

\subsubsection*{Procedure}
In the case of a numerical label, we;

\begin{itemize}
	\item Build a single leaf node, that is the average of the classified numerical value
	\item Reconstruct the tabular data and create the {\em pseudo residual}
	\item Build a tree that predicts the residual
\end{itemize}

So the idea is, we start with the original prediction and then "add" the newly constructed tree.

\subsubsection*{Learning Rate}
For each addition of tree we add a learning rate quantifier, this tackles the problem of over fitting. Multiplier from 0 to 1 obviously.


\subsubsection*{Pre-requisite; Regularization}



\end{document}
