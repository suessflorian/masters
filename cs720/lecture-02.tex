\documentclass{article}
\usepackage{mathtools}
\usepackage{amsmath}
\DeclarePairedDelimiter\floor{\lfloor}{\rfloor}
\begin{document}
(1st March)
\section*{Finishing up Permutations with Unranking}
To unrank $P_j$ for permutation of length $j$ Let $P_{j-1} = \floor*{P_j/j}$ and remainder $R_{j-1} = P_j \text{ mod } j$. That is $P_j = P_{j-1}+R_{j-1}$.

Compute $\pi_{j-1} =  unrank(P_{j-1})$... 
If $P_{j-1}$ even then $\pi_j = \pi_{j-1}$ with "$j$" inserted at the position $R_j-1$ from the right.
If $P_{j-1}$ odd then $\pi_j = \pi_{j-1}$ with "$j$" inserted at the position $R_j-1$ from the left.

\subsection*{Example Unrank(19)}
Table makes more sense, but cbf'd writing one out sorry...working down... we see;
\begin{align}
j=4, Unrank(19),P_{j-1} &= 4 , R_{j-1} = 3,\pi_j = ?, direction = ?\\
j=3, Unrank(4), P_{j-1} &= 1 , R_{j-1} = 1,\pi_j = ?, direction = ?\\
j=2, Unrank(1), P_{j-1} &= 0 , R_{j-1} = 1,\pi_j = ?, direction = ?\\
j=1, Unrank(0), P_{j-1} &= 0 , R_{j-1} = 0,\pi_j = ?, direction = ?
\end{align}

And then working back up from our starting permutation of "1". Note direction is derived on {\bf current} $P_{j-1}$ and position is based on {\bf current} $R_{j-1}$.

\begin{align}
j=4, Unrank(19),P_{j-1} &= 4 , R_{j-1} = 3,\pi_j = 4231, \leftarrow\\
j=3, Unrank(4), P_{j-1} &= 1 , R_{j-1} = 1,\pi_j = 231, \rightarrow \\
j=2, Unrank(1), P_{j-1} &= 0 , R_{j-1} = 1,\pi_j = 21, \leftarrow\\
j=1, Unrank(0), P_{j-1} &= 0 , R_{j-1} = 0,\pi_j = 1
\end{align}

\section*{Subset}
So emphasis is on lack of order importance, ie; $\{1, 2\} = \{2, 1\}$. We dive right into binomial coefficients $\binom{n}{k} = \frac{n!}{k!(n-k)!} = \frac{n_k}{k!}$.

Some facts ({\em based of really cool binary string mapping of possible subsets});
\begin{itemize}
  \item $\sum_{k=0}^{n}{\binom{n}{k}} = 2^n$, basically the number of subsets possible for a set of objects size $n$, is $2^n$.
  \item $\binom{n}{k} = \binom{n}{n-k}$
  \item $\binom{n}{k} = \binom{n-1}{k-1} + \binom{n-1}{k}$
\end{itemize}

\subsubsection*{Example $\binom{5}{3}$}
\begin{math}
|\{(123),(124),(125),(134),(135),(145),(234),(235),(245),(345)\}|\\
= 10 = \frac{5!}{3!(5-3)!}
\end{math}

\subsubsection*{Co-lexicographic Ordering (Colex)}
So how we just listed the above, we opted for a lexicographic ordering for sake of nicety above... Right to left (while keeping confusing for representation by still keeping the set order left to right).
\begin{align}
(123) \nonumber \\ 
(124) \nonumber \\
(134) \nonumber \\ 
(234) \nonumber \\
(125) \nonumber \\
(135) \nonumber \\ 
(235) \nonumber \\ 
(145) \nonumber \\ 
(245) \nonumber \\ 
(345) \nonumber \\ 
\end{align}

Okay that took a while to get right mentally...



\end{document}
