\documentclass{article}
\usepackage{tikz}
\usepackage{amsmath}

\begin{document}

(2nd March)
\section*{Integer Paritions of $I_n$}
All $k$ tuples $\lambda = (\lambda_1 \lambda_2 ... \lambda_k)$ where $\lambda_1 \geq \lambda_2 ... \geq \lambda_k$ and $\lambda_1+\lambda_2+\lambda_k = n$. (decreasing).

"superscript/multiplicitive" notation 
\begin{equation}
\lambda = \lambda_{1}^{p_1}\lambda_{2}^{p_2}...\lambda_{m}^{p_m}
\end{equation}
eg; $\lambda=(5,5,5,3,2,2,1,1)$ can be expressed as $5^{2}32^{2}1^{3}$.

Then there is also the pictorical representation... imagine a 2D plane. The number of blocks in this representation adds to $n$. And in decreasing order from top to bottom you'd lay out horizontally the blocks quantities for each parition of $n$. This motivates the idea of a conjugate partition $\lambda'$. (bijective).

\section*{Theorem}
The number of paritions of $n$ with $k$ parts is equal to the number of paritions of $n$ with the greatest part equal to $k$.

{\em visually proven using the conjugate mapping shown via the pictorial representation}

\section*{Reverse Lexicographic Order}
Lexic, Colex, Reverse Lex are the orderings covered so far.

For $n=4$, example of "Reverse Lex".
\begin{align*}
4 \\
3 1 \\
2 1 1 \\
1 1 1 1
\end{align*}

For $n=5$, example of "Reverse Lex".
\begin{align*}
5 \\
4 1 \\
3 2 \\
3 1 1 \\
2 2 1 \\
2 1 1 1 \\
1 1 1 1 1 \\
\end{align*}

Michael mentioned a successor function here, we should code it up (it is now).

\end{document}
