\documentclass{article}
\usepackage{amsmath}
\usepackage{bm}
\title{COMPSCI 750: Computation Complexity \\ Second Assignment}
\author{Florian Suess}
\begin{document}
\maketitle

\section*{Computable Transformation}
We dive into the inards of the \emph{reverse string} function $\bm{R}$. Consider the very nifty recursive definition;
\begin{align*}
	\text{for } a \in \{0,1\} \text{ then } \bm{R}(a) = a \\
	\text{let } w \in \{0,1\}^* \text{ then } \bm{R}(wa) = a\bm{R}(w)
\end{align*}

\subsubsection*{Lemma 1: $\bm{R}(\bm{R}(s)) = s$ for all $s \in \{0,1\}^*$.}
Inductively then, give some $l \in \{0,1\}$, then by definition we have $\bm{R}(l) = l$. Let $w \in \{0,1\}^*$ be something arbitrary, we now assume that for any $x \in \{0,1\}^*$ such that $|x| < |w|$, we have $\bm{R}(\bm{R}(x)) = x$. Now for this $w$, it must be the case that $w = xl$ for some other $l \in \{0,1\}$ and $x \in \{0,1\}^*$. Clearly $|x| < |w|$.

\begin{align*}
	\bm{R}(\bm{R}(w)) = \bm{R}(\bm{R}(xl)) = \bm{R}(l\bm{R}(x)) = \bm{R}(\bm{R}(x))l = xl = w
\end{align*}

\subsubsection*{$\bm{R}$ is a bijection}
For \textbf{injectivity}, let's assume that for some $a,b \in {0,1}^*$ that $\bm{R}(a) = \bm{R}(b)$. By lemma 1, applying $\bm{R}$ both sides yields $a = b$. For \textbf{surjectivity}, given some $a \in \{0,1\}^*$, we can construct $\bm{R}(a) \in \{0,1\}^*$ such that $\bm{R}(\bm{R}(a)) = a$ by lemma 1.

\section*{Research Question}
\section*{Mathematical modelling}

\end{document}
