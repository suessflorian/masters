\documentclass{article}
\setlength{\parskip}{\baselineskip}%
\setlength{\parindent}{0pt}%
\usepackage{algorithm}
\usepackage{algorithmic}
\usepackage{amsmath}
\usepackage{amssymb}
\title{Assignment}
\author{Florian Suess}
\begin{document}
\maketitle
\section{Program Analysis}
\subsection{Deciding if an even integer N is the sum of two primes}
For clarity further along - we begin with a simple utility \textbf{isPrime} that via primitive trial division method determines if an input candidate number $C$ is prime.

\begin{algorithmic}[1]
	\REQUIRE $C \geq 0$
	\IF{$C = 1 \text{ \textbf{or} } C = 2$}
		\RETURN 1
	\ENDIF
	\STATE $f \leftarrow C - 1$
	\WHILE{$f > 1$}
		\IF{$f \mid C$}
				\RETURN 1
		\ENDIF
		\STATE $f \leftarrow f - 1$
	\ENDWHILE
	\RETURN 0
\end{algorithmic}

If it is not clear, we \textbf{return} to indicate the returning of a value to a potential caller. Suppose this algorithm runs on it's own, we interpret \textbf{return} as a \textbf{print} followed by a \textbf{stop}.

\pagebreak

\subsubsection*{Is Goldbach number algorithm}
\begin{algorithmic}[1]
	\REQUIRE $N > 2 \text{ \textbf{and} } 2 \nmid N$
	\STATE $c \leftarrow N - 1$
	\WHILE{$c > 2$}
		\IF{\textbf{isPrime(c)}}
			\STATE $c' \leftarrow N - c$
			\IF{\textbf{isPrime(c')}}
				\RETURN 1
			\ENDIF
		\ENDIF
		\STATE $c \leftarrow c - 1$
	\ENDWHILE
	\RETURN 0
\end{algorithmic}


\subsection{Does the above halt?}
We shall first prove that for any $C > 0$, \textbf{isPrime} halts. It's tediously obvious as $f$ is clearly monotonically decreasing towards a bound of 1. It will inevitably reach this bound and exit the only loop then halt, if not halting beforehand.

We use this result of \textbf{isPrime} directly for the main algorithm. Similarly to above $c$ is monotonically decreasing towards a bound of 2. It will inevitably reach this bound and exit the only loop and halt, if not halting beforehand.

\subsection{Does the proposed algorithm in the assignment solve the Goldbach conjecture?}
Yes. I'm heading to Faber and Faber to pick up my cool million dollars. \textbf{I am ofcourse kidding}.

\pagebreak
\section{Research Question}
In reference to John Baez; Computing the uncomputable blog post... a direct summation of Joel David Hamkins (and friends like Woodin) result.

\subsection{The main result}
There is \textbf{a} Turing machine $T$ that when given any function $f: N \rightarrow N$, there is a model of Peano Axioms (PA), that is, the triple $(N, 0, S)$, $0 \in N$ and $S: N \rightarrow N$ that in tandem satisfy the PA, such that in this model, if we give $T$ any \emph{standard} natural $n$ as input, it halts and outputs $f(n)$.

\subsection{My interpretation of Joel David Hamkins proof}
\subsubsection{Preliminaries}
I personally will focus within the context of Peano Axiom (PA) as I'm struggling enough with an understanding. A model of PA is a mere triple $(N, 0, S)$, where $N$ is a set, $0 \in N$ and $S: N \rightarrow N$ satisfies the PA the precise "structure" of $N$ and $S$ is up to interpretation (although it is worth noting that $N$ is necessarily infinite). We are all aware of the "standard" PA model, and so I won't dwell. What we need to do is speak about "non-standard" models of PA - construction originally driven by the works of Löwenheim–Skolem. The existence of such models are initial segment isomorphic to the "standard" natural numbers. Initial segment isomorphism suggests the existence of an \emph{upper set} of elements that, as John Baez puts it, are "tacked" on the end of the set of "standard" natural numbers. Somewhat concretely I view these "non-standard" numbers as having the property that they are strictly greater than $0$ and every element of the model you can reach from 0 by applying $S$ (\textbf{finitely} many times). However, these elements still obey all the axioms of PA (which nessecerailly pushes an infinite size of this upper set also).

John Baez points out that a sentence $p$ in such a model is merely provable, or not provable, and factually speaking it either holds or doesn't hold - importantly \textbf{relative} to the adopted model we're using to interpret $p$.

The proof described in the original Hamkins blog derives \textbf{a single} Turing machine $T$ uses an infinite sequence of Rosser sentences and their negations that depend on the given $f$ to actually "build" a model of PA that allows this Turing machine $T$ to computably express this function $f$ on the standard input domain. By that I mean, for any standard $n$ input, this Turing machine halts and computes $f(n)$.

So... it is then proven that if we were to define a halting decider function using Hamkins (and friends)

\subsection{Does the main result contradict the Halting Theorem?}


\section{Mathematical Modelling}
\subsection{Monochromatic arithmetic progressions}
First, we define concretely what it means for an "infinite binary sequence contains a monochromatic arithmetic progression of \emph{any} length". 

Given a infinite binary sequence $s = s_{1}s_{2}s_{3}...$ and some $k>0$ we say there $\exists i,t \geq 1$ such that $s_{i}s_{i+t}s_{i+2t}...s_{i+(k-1)t}$ is $\in \{1^k,0^k\}$.

\subsection{Does every infinite sequence contain a monochromatic arithmetic progression?}
A pretty vacuous application of the \textbf{finite Van der Waerden theorem}. Give me your infinite binary sequence $s$ and desired $k$ value. In close relation to the theorem delivery, fix $c = 2$ (colouring). By thereom, there $\exists \gamma$ such that the prefix of length $\gamma + 1$ has a monochromatic arithmetic progression of length $k$.

\subsection{And for ternary sequences?} 
Given a infinite ternary sequence $s = s_{1}s_{2}s_{3}...$ and some $k>0$ we say there $\exists i,t \geq 1$ such that $s_{i}s_{i+t}s_{i+2t}...s_{i+(k-1)t}$ is $\in \{1^k,2^k,0^k\}$.

Verbatim the exact same proof applied (heavy reliance on the finite Van der Waerden theorem), where we now fix $c = 3$.

\section{Scratch References}
\begin{itemize}
	\item trial division for algorithm
\end{itemize}
\end{document}

