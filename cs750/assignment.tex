\documentclass{article}
\setlength{\parskip}{\baselineskip}%
\setlength{\parindent}{0pt}%
\usepackage{algorithm}
\usepackage{algorithmic}
\usepackage{amsmath}
\usepackage{amssymb}
\title{Assignment}
\author{Florian Suess}
\begin{document}
\maketitle
\section{Program Analysis}
\subsection{Deciding if an even integer N is the sum of two primes}
For clarity further along - we begin with a simple utility \textbf{isPrime} that via primitive trial division method determines if an input candidate number $C$ is prime.

\begin{algorithmic}[1]
	\REQUIRE $C \geq 0$
	\IF{$C = 1 \text{ \textbf{or} } C = 2$}
		\RETURN 1
	\ENDIF
	\STATE $f \leftarrow C - 1$
	\WHILE{$f > 1$}
		\IF{$f \mid C$}
				\RETURN 1
		\ENDIF
		\STATE $f \leftarrow f - 1$
	\ENDWHILE
	\RETURN 0
\end{algorithmic}

If it is not clear, we \textbf{return} to indicate the returning of a value to a potential caller. Suppose this algorithm runs on it's own, we interpret \textbf{return} as a \textbf{print} followed by a \textbf{stop}.

\pagebreak

\subsubsection*{Is Goldbach number algorithm}
\begin{algorithmic}[1]
	\REQUIRE $N > 2 \text{ \textbf{and} } 2 \nmid N$
	\STATE $c \leftarrow N - 1$
	\WHILE{$c > 2$}
		\IF{\textbf{isPrime(c)}}
			\STATE $c' \leftarrow N - c$
			\IF{\textbf{isPrime(c')}}
				\RETURN 1
			\ENDIF
		\ENDIF
		\STATE $c \leftarrow c - 1$
	\ENDWHILE
	\RETURN 0
\end{algorithmic}


\subsection{Does the above halt?}
We shall first prove that for any $C > 0$, \textbf{isPrime} halts. It's tediously obvious as $f$ is clearly monotonically decreasing towards a bound of 1. It will inevitably reach this bound and exit the only loop then halt, if not halting beforehand.

We use this result of \textbf{isPrime} directly for the main algorithm. Similarly to above $c$ is monotonically decreasing towards a bound of 2. It will inevitably reach this bound and exit the only loop and halt, if not halting beforehand.

\subsection{Does the proposed algorithm in the assignment solve the Goldbach conjecture?}
Yes. I'm heading to Faber and Faber to pick up my cool million dollars. \textbf{I am ofcourse kidding}.



\section{Research Question}
\section{Mathematical Modelling}
\subsection{}
Define the property "the infinite binary sequence contains a monochromatic arithmetic progression".

Given a infinite binary sequence $s = s_{1}s_{2}s_{3}...$
\begin{math}
	t>0,k>1; \exists
\end{math}

\subsection{}
Does every infinite sequence contain a monochromatic arithmetic progression?

\subsection{} How about for ternary sequences?

\section{Scratch References}
\begin{itemize}
	\item trial division for algorithm
\end{itemize}
\end{document}

