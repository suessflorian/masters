\documentclass{article}
\usepackage{algorithmic}
\usepackage{amsmath}
\title{Assignment}
\author{Florian Suess}
\begin{document}
\maketitle
\section{Program Analysis}
\subsection{}
Keeping it simple with a primitive trial division method. We consider all candidate factors $f < N$ and check divisibility;

\begin{algorithmic}[1]
	\REQUIRE $N > 2$
	\STATE $f \leftarrow 2$
	\WHILE{$f < N$}
		\IF{$f \mid N$}
			\RETURN 1
		\ENDIF
		\STATE $c \leftarrow c + 1$
	\ENDWHILE
	\RETURN 0
\end{algorithmic}

If it is not clear, we \textbf{RETURN} to indicate the returning of a value to a potential caller (proactively due to the following question). Suppose this algorithm runs on it's own, we interpret \textbf{RETURN} as a \textbf{print} followed by a \textbf{stop}.

\section{Research Question}
\section{Mathematical Modelling}
\subsection{}
Define the property "the infinite binary sequence contains a monochromatic arithmetic progression".

Given a infinite binary sequence $s = s_{1}s_{2}s_{3}...$
\begin{math}
	t>0,k>1; \exists
\end{math}

\subsection{}
Does every infinite sequence contain a monochromatic arithmetic progression?

\subsection{} How about for ternary sequences?

\section{Scratch References}
\begin{itemize}
	\item trial division for algorithm
\end{itemize}
\end{document}

