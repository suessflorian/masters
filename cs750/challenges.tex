\documentclass{article}

\newenvironment{answered}{\par\normalfont}{}
\title {Challenges}
\author {Florian Suess}
\begin{document}
\maketitle
\section{Loops and non-halting programs}

\begin{enumerate}
	\item A program containing a loop never halts and deciding if a program contains a loop is easy. Why?
	\begin{answered}
		As brought to light in slide 11, suppose a program has a line numbered its set of instructions. If we define the sequence of some program $P$ and input $v$, say $S(P, v)$ as the execution flow of the program in terms of the line number of the current executed instruction line number. We can identify if after some $n$th item in the sequence of $S(P,v)$ contains nothing but the repetive pattern (loops) of instructions $(x_1, x_2, ..., x_i)$ where $i$ simply represents the size of this loop.
	\end{answered}
\item Show that the program below does not halt;
\begin{verbatim}
 1 i=1
 2 while i >= 1 do 
 3   j = 1
 4   while j <= i do
 5     print 0
 6     j = j + 1
 7   end while	
 8   i = i + 1
 9 end while
10 stop
\end{verbatim}
 \begin{answered}
	 Define the sequence $S(i)$ as the of values set to $i$ as the program runs. As defined in line 1, it begins with 1. Filtering the instructions of the program to only changes to variable $i$... it's clear that $\forall n > 0, S(i)_n \geq 1$. Hence the program will never satisfy any condition other than $i \geq 1$, meaning the condition on line 2 will never fail thus never exiting the loop.
 \end{answered}
\end{enumerate}


\end{document}
