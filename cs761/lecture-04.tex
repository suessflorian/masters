\documentclass{article}
\usepackage{datetime}
\begin{document}
\title{Informed Search Part One}
\newdate{date}{25}{07}{2022}
\date{\displaydate{date}}
\author{Florian Suess}
\maketitle

A light recap on the older stuff, \textbf{search problems, search tree, search algorithms}.

\section{Informed Search Part 1}
\section{Cost Sensitive Search}
DFS is not optimal... BFS finds the shortest path in terms of number of actions. Not the least-cost path hence we looked at Uniform Cost Search (priority queue implementation).

The issue we will talk about today is how this algorithm explores options \textbf{in every direction}. No information about the goal location.

\subsection{Informed Search Heuristic}
A heuristic is:

\begin{itemize}
	\item function that \textbf{estimates} closeness
	\item not general to all problems (specific to a problem)
	\item examples; "Manhattan" distance, "Euclidean" distance
\end{itemize}

And that's kinda the whole crux of informed vs uninformed search algorithms... one uses heuristics, the other does not.

\begin{itemize}
	\item Generally not complete. Can get stuck in loops, but complete in finite space with repeated-state checking.
	\item time to search $O(bm)$, (branching factor, maximum depth).
	\item space $O(bm)$, (all in memory).
	\item optimal? No?
\end{itemize}

\section{Summary}
Heuristic's can be applied to reduce cost. Greedy search tries to minimise cost from current node $n$ to the goal. Informed search makes use of problem-specific knowledge to guide progress of search.

\end{document}
