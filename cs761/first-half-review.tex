\documentclass{article}
\usepackage{outlines}
\usepackage{datetime}
\usepackage{amsmath}


\begin{document}
\title{Finals Prep}
\newdate{date}{26}{07}{2022}
\date{\displaydate{date}}
\author{Florian Suess}
\maketitle

\section{PAGE Description}
Used to group similar tasks together. An axis that describes key characteristics of a task to be performed. 

\begin{itemize}
	\item Percepts
	\item Actions
	\item Goal
	\item Environment
\end{itemize}

\section{PEAS Model}
Used to group similar agents together. An axis that describes key characteristics of an agent.

\begin{itemize}
	\item Performance Measure
	\item Environment
	\item Actuators
	\item Sensors
\end{itemize}

\section{Environment (eight) types}
\begin{itemize}
	\item Stochastic vs. Deterministic
	\item Continuous vs. Discrete
	\item Situated vs. Simulated
	\item Single Agent vs. Multiple Agent
	\item Fully Observable vs. Partially Observable
	\item Static vs. Dynamic

	\item Known vs. Unknown
	\item Episodic vs. Sequential (Functional vs. Stateful)
\end{itemize}

\section{State Space Problems}
A set of states, subset of which contains the starting states. We have a set of actions. A successor function that takes in a state and action and returns a new state. We have a goal function that given a state returns true or false (which includes a criterion of acceptable solutions).

\subsection{Specify the state space problem behind a Rubik's cube}
\begin{itemize}
	\item Each cube, $c_{i} = (x_{i}, y_{i}, z_{i}, ry_{i}, rx_{i})$ where $(x_{i},y_{i},z_{i})$ represents the position of the cube, and $ry_{i} \in {0, 90, 180}$ representing the rotation through the y-axis, and $rx_{i} \in {0, 90}$ representing the rotation through the x-axis.
	\item The set of states $K$, some $k \in K$ appears as $\{c_1, c_2, ..., c_n\}$ where $n=27$.
	\item Set of actions $A = \{ a^u_{\text{rotate right}}, a^d_{\text{rotate down}}, a^r_{\text{face clockwise}} \}$
	\item Successor function $f: (K, A) \rightarrow K$
	\item We can choose an arbitrary goal state, although if we were to color each cube $c_i$, a nice goal state would be the reserving of $c_1 = (1,1,1,0,0), c_2 = (2, 1, 1, 0, 0), c_3 = (3, 1, 1, 0, 0), c_4 = (1, 2, 1, 0, 0), ...$ such that each this state represents a cube with the same color sides. There will be 6 goal states. 1 for each rotation of the entire cube.
\end{itemize}

\section{Search Problems: Search Tree}
A tree super imposted onto the state space graph representation of a state space problem. Root being reserved for the start state.

\section{Uninformed Search}
There's a few we covered \textbf{Breadth-FS, Depth-FS, Uniform Cost Search, Depth Limited Search, Iterative Deepening Search, Bi-Directional Search}

\section{Informed Search}
There's a few we covered \textbf{Greedy-FS, A*, Iterative Deepening A*}

\subsection{Heuristics}
All informed searches rely on heuristic functions (take $h$) to effectively guide the search towards the right direction. Two properties of interest are $admissability$ and $consistency$.

\begin{itemize}
	\item \textbf{admissibility}; given some node $n$, $h(n) \leq a(n)$ where $a$ represents the actual cost.
	\item \textbf{consistency}; given some node $n$ and it's successor $n'$, $h(n) \geq h(n') + c(n, n')$ where $c$ represents the cost between two adjacent nodes.
\end{itemize}

\end{document}
