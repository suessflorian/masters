\documentclass{article}
\usepackage{outlines}
\usepackage{csquotes}
\begin{document}

Lecture times are a little awkward. Tutorials though are from 5pm-6pm has a physical room.

\section{Course Objectives}
\begin{itemize}
				\item Problem solving: search, heuristic, constraint satisfaction, reasoning
				\item Representation: search space, knowledge, semantics 
				\item Uncertainty: lack of knowledge, probability
\end{itemize}

\subsection{Content}
Where throughout the class, we will have multiple lectures.
\begin{itemize}
				\item Artificial Intelligence
				\item Agents and Environments
				\item Search
				\item Constraint Satisfaction
				\item Logic
				\item Planning
				\item Reasoning with Uncertainty
				\item Natural Language Processing
\end{itemize}

\subsection*{Textbooks}
Couple recommended textbooks... should buy! Standard books for AI course.

\subsection*{Assessment}
\begin{outline}
				\1 Assignments 40\%
								\2 20\% Assignment, week 1-6, August 19th
								\2 10\% Assignment, week 6-7, Sept 30th
								\2 10\% Assignment, week 10-12, Nov 11th
				\1 Mid-semester exam 30\%
								\2 Open book, online, week 1-6, August 26th
				\1 Final 30\%
								\2 Open book, online, week 7-12, TBD
\end{outline}

\section{First lecture}
Here we go with the lecturers perspective of the history and future. There is a long history, since 428BC or whatever... however the evolution of definition of AI evolving is of interest.

\begin{itemize}
				\item Thinking humanly vs acting rationally
				\item Acting humanly vs acting rationally
\end{itemize}

Acting humanly; development of the turing test. Alan Turing, was designed as a thought experiment that would sidestep the philosophical vagueness question "can a machine think?". A computer passes the test if a human interrogator, after posing some writen questions, cannot tell wether the writen response come from a person or a machine. (Notably the turing test was first passed in 2014! Simulation 13yr old Ukranian boy, albeit lecturer believes optimised for deception).

We look at example dialogues of above.

\section{Acting humanely}

\subsection*{Google LaMDA}
Extract of recorded dialogue that shows \emph{believability}.

\subsection*{Winograd Schema Challenge}
Specific focus on applying resolutions of anaphora. Supposedly easy for humans, difficult for machines.

\subsection*{Chinese Room Argument}
First published in the 80s. Imagine a person who doesn't speak Chinese standing in a room with a set of instructions on how to translate Chinese characters. They receive a paper with a Chinese writing, then translate it using the instructions, and give back the translation.

Question: Do they \emph{understand} Chinese?

\subsection{Conclusive thoughts on these cases}
Perceived intelligence from the outside, doesn't mean there is an understanding.

\section{Thinking humanely}
Cognitive science brings together computer models from AI and experimental techniques from psychology. Reflecting on thoughts, psychy tests, observe brain activity (setback being the complexity of a human brain).

\subsection*{Thinking like an animal}
Hermaphrodite nematode has only 302 neorons, 132 muscles and 26 non-muscle end organs. Multiplying is exact copying. Super simple animal we can use to progress from (2019 was published the complete connectoid published, virtualised this entire animal).

\subsection{Thinking Rationally}
A \emph{syllogism} is logical structure that applies deductive reasoning to arrive at a conclusion given true premises.

\textquote{Socrates is a man; all man are mortal therefore Socrates is mortal.}

The logic-based programs in theory can solve any solvable problem described in logical notation, \textbf{but} some knowledge is hard to state in formal terms and in practice solving a problem can require too much computational resources.

\subsection{Acting Rationally}
\emph{This course will be using this definition closely.}

\blockquote The term AI contains an explicit reference to the notion of intelligence. However, since intelligence (both in machines and in humans) is \textbf{vague} concept, although it has been studied at length by psychologoists, biolgists, and neorscientists, AI researchs use mostly the notion \textbf{rationality}. This refers to the ability to choose \textbf{the best action} to take in order to achieve a cetain \textbf{goal}, given certain \textbf{criteria} to be optimied and the avialable resources. Ofcourse, rationality is not the only ingredient in the concept of intelligence, but it is a significant part of it.

\textbf{AI} is the synthesis and analysis of \textbf{computational agents} that \textbf{act intelligently}. And we say that an agent acts intelligently if:
\begin{itemize}
				\item its actions are appropriate for its goals and circumstances
				\item it is flexible to changing environments and goals
				\item it learns from experience
\end{itemize}

See examples!

\section{School of thought}
Two schools
\subsection{Weak AI}
Belief that machines can be made to act as if they were intelligent, limited to a narrow area, specific to a task.
\subsection{Strong AI}
Belief that those machines are \emph{actually} thinking which allows "under development"; can perform any task a human can do (eventually).

\end{document}
