\documentclass{article}
\usepackage{datetime}
\usepackage{outlines}
\begin{document}
\title{Part II: Logic}
\newdate{date}{12}{09}{2022}
\date{\displaydate{date}}
\author{Florian Suess}
\maketitle

Weeks 7-9 will be spent on logic, taught by Yang Chen.

\section{What's Logic}
We start with a motivational question, prompting the use of our ominous inference. Historical evolving of Aristotle, Yin-Yang's idea of logic. Modern in the 19th-20th centuries, probably all not important.

Since we are going to be writing prolog (an inference based logical language?) the key mantra is inferring the answers to propositions using a knowledge base. The inference technique we will talk about I guess.

\subsection{Types}
\begin{itemize}
	\item Propositional Logic
	\item First-Order Logic
\end{itemize}

\section{Agents and World Models}
Review, but in perspective of logic. A \textbf{computational agent} is an entity capable of;

\begin{itemize}
	\item perceiving the environment
	\item deliberating
	\item performing actions
\end{itemize}

\subsection{PEAS Model; Agent Task}
\begin{itemize}
	\item Performance Measure; evaluation function essentially, (task state) $\rightarrow$ rating
	\item Actuators; things that "can be done" to modify the task state.
	\item Sensors; ways of understanding the task state
	\item Environment; the set of task states
\end{itemize}

\textbf{percept} is the agents perceptual input at any given instance. An agents \textbf{percept sequence} is the complete history of everything that the agent has perceived, and finally, \textbf{agent program} maps a given percept sequence to an action.

So kinda like a function based computational agent, we saw this early on in lectures. Now a \textbf{world model} encodes information about... the world. Specifically two sides, how the world evolves without provocation, and how the world evolves with provocation.

A \textbf{model based agent program} uses a world model to determine the agent's action. Maintains internal state, updated via \textbf{state-transition function}. Also note an agents behaviours may be directed by goals or utilities (goal based agents).

\section{Representing the World}

\subsection{Atomic Representation}
Explicitly represent each state of the world and their transitions. Kinda like a FSM diagram.

\subsection{Factored Representation}
A state of the world is captured by a collection of variables and values.

\subsection{Structured Representation}
Simply keeping the variable values is often insufficient, an agent may only have \emph{partial knowledge} about the state; variables may correlate with each other through constraints. \textbf{Potential for reasoning}.

A structured representation then is a world model that describes the world using variables, while capturing \textbf{knowledge/constraints} and then hosting some sort of \textbf{inference engine (reasoning)} to derive knowledge from percept sequences.

\emph{Note: we go through a "wumpus" example, loved the rigorous definitions of "knowledge" via constraints given... should review if in doubt. Inference is shown via some sort of process of elimination of square classifications on visit.}

\end{document}
